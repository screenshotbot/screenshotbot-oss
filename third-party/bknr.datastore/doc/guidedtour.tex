\chapter{Guided Tour}

\vbox{
\centering
\includegraphics{guidedtouricon}
\vspace{1cm}
}

Let's take a guided tour of the BKNR sourcecode. Most facilities are
independent of each other, and can be used to build completely
different types of applications. However, all use the common toolbox
called `BKNR-UTILS', which contains a lot of small lisp
functions. These functions are not very important, and we will leave
them behind to get to the first important facility of the BKNR launch
platform.

\section{The Datastore}

The datastore is one of the oldest facilities in BKNR, its role is to
store the application data, so that this data is persistent between
LISP sessions. Most existing applications use a back-end
database to store persistent objects, for example a relational
database or an object-oriented database. Despite the immediate
advantages of having an existing database (reliability, speed,
management tools), this approach is quite cumbersome: every
application data structure has to be mapped onto a data structure in
the database. This is not trivial, as can be seen by the gazillions
attempts to write automatic conversion tools. Another fundamental
problem is the integration of the database into the language itself,
and into the development workflow. Errors in the database are
"external" to the LISP process, rollback, error-catching, transactions
are not as easy to write as they would be by having an "internal"
database. In the end, we believe that the advantages of having an
external database are outweighed by the disadvantages.

With the decreasing hardware costs nowadays, it is quite feasible for
most applications to keep their entire dataset in memory, and indeed
this is the approach taken by the BKNR datastore. Even with a growing
data set, memory costs decrease faster than the data grows, so that
evolution is not a problem. This approach is known as "prevalence",
and the BKNR datastore was initially based on the prevalence solution
by Sven Van Caekenberghe (also see
`http://homepage.mac.com/svc/prevalence/readme.html'). The key points
of the prevalence model are:

\begin{itemize}
\item All data is held in RAM

\item Changes to persistent data are written to a transaction log
   file. When loading the datastore, the transaction log is read, and
   all changes are applied to the persistent data. All changes ever
   made to the persistent data are executed in order, and the data is
   recovered.

\item If the data model supports it, the persistent data state can be
   captured and written to a file, which is called a "snapshot
   file". The snapshot file is read when loading the datastore, and
   the snapshotted data is used as a "starting point" for the
   transaction log.
\end{itemize}

In the datastore, transactions which are logged to the transaction log
are LISP functions defined using the `DEFTRANSACTION' form. Thus, all
transactions transforming persistent data are made explicit in the
source code. This if different from object-oriented databases, where
the fundamental transactions are object creation, object deletion and
slot access, which are not special cases in the prevalence model at
all. The main problem with this approach is that it is possible to
modify the persistent data in a way that is not logged into the
transaction log, which leads to lost data after reloading the
database. However, the BKNR datastore has a lot of development and
debugging helps, warning the user when he is making dangerous changes
(or forbidding them alltogether).

The datastore has gone through a few development iterations, which are
actually quite interesting to explain, as they show which compromises
have been made, and how the datastore can be tweaked. After having
used the prevalence solution of Sven van Caekenberghe and deciding
that it didn't quite fit our needs, we wrote a simple datastore
closely modelled on Sven's approach. However, the datastore featured
helpers to create and modify indexed CLOS objects. Also, the
transaction log consisted of SEXPs, and the loading of the transaction
log consisted of using the `LOAD' LISP function. The main part of the
object datastore consisted of a really big `DEFINE-PERSISTENT-CLASS'
macro, which generated a `DEFCLASS' form, and overrided a lot of
methods, making up a store generic method protocol. Special slot
options could be used to tell the store to index objects. For example,
you could specify that the slot `NAME' of the class `USER' was stored
in an index called `USER-NAME-INDEX'. The persistent objects could be
snapshotted and written to a snapshot file, which again was a LISP
file.

The main problem with this first approach was the lack of real CLOS
support. For example, you could modify an object using standard CLOS
methods like `SETF SLOT-VALUE' without ever getting a warning that
modifying the persistent state outside a transaction is
dangerous. Moreover, the `DEFINE-PERSISTENT-CLASS' macro was difficult
to modify, and the slot options for indices could not be modified
easily. This lead to the development of a separate index layer, a
separate transaction system, and of an object datastore combining the
two systems with additional CLOS support built using the Metaobject
Protocol.

\subsection{The transaction layer}

The transaction layer is now separate from the object oriented
datastore. It provides the functionality to declare explicit
transactions (still using the `DEFTRANSACTION' form), and to serialize
the transactions coming from multiple threads to a transaction log in
the filesystem. Using the transaction system is done by instantiating
an instance of the `STORE' class (or `MP-STORE' to serialize
concurrent transactions) and specifying a directory where the
transaction logs has to be stored. Only a single transaction system
can be open in a LISP session. When a transaction is executed, the
store serializes the transaction call (the transaction name and its
arguments) to the transaction log. It also stores the timestamp at
which the transaction was executed. This allows the user to restore
the transaction system state until a certain time (for debugging his
application or reverting to an old data state). When a datastore is
created or restored, the transaction log is read, and all the
transaction calls are re-executed. After all the transactions have
been executed, the persistent state is restored.

A transaction store can have multiple subsystems, which control a
certain subset of the persistent data. For example, the object
datastore which controls persistent CLOS objects is realized as a
subsystem of the transaction layer. A subsystem can snapshot the
current persistent state, and write it to disk. When the persistent
data is snapshotted, the transaction log can be discarded. Restoring
this state then consists of loading the snapshot file, and then
executing the transactions stored in the new transaction log. This
allows for a much faster restore procedure. The snapshotting procedure
takes care to backup the old transaction log, so that a restore until
a certain timestamp is still possible.

\subsection{The object datastore}

The object datastore combines several aspects:
\begin{itemize}
  \item It is a subsystem of the transaction layer, and can snapshot
persistent CLOS objects to a snapshot file, saving their slot values.
  \item It provides a CLOS Metaclass for persistent objects. Using
this metaclass, you can specify additional slot options for persistent
objects. For example, you can specify that a certain slot is
transient, which means that it will not be snapshotted and that its
value can be changed outside of a transaction.
  \item The persistent object metaclass also provides the index
facility of `BKNR-INDICES', so that persistent objects automatically
get indexed when restored from the store. The index facility is also
used by the store itself to keep track of the instantiated objects of
each class, and to index the objects by their unique ID.
\end{itemize}

The object datastore also provides a few helpers for developers. For
example, you cannot change the value of a slot when you are outside of
a transaction. The store throws an error when such an illegal
operation is made. The store also warns you when you change a class
definition. As the class definitions are not stored in the datastore,
a snapshot is necessary after schema evolution.

\section{The XML Import/Export Facility}

BKNR applications often have to communicate with the external world,
for example getting data from external sources, or exporting the data
from the object datastore to external sources. In order to reduce the
amount of conversion and parsing code that has to be written to
accomodate these tasks, the BKNR framework provides an automatic
mapping from CLOS classes to XML data. Using a special Metaclass for
XML objects, you can map a CLOS class to XML according to a XML
DTD.

In the following paragraphs, we will use the CLOS class `USER' to show
how the XML import/export facility works. The class `USER' has a slot
`PARENT' which points to another `USER', a slot `NAME' which contains
a string (the name of the user, obviously) and a slot `AGE' which
contains an integer (the age of the user).

A DTD consists of element declaration and of attribute declarations
(we will leave entity declarations aside for now). Most of the time,
the mapping between the XML data and the CLOS representation is quite
straightforward. An object is represented as an XML element, and its
slots are represented as either attributes of the element, or child
elements. All the data in the XML file is encoded as a string, and
read as such by the CXML parser used by BKNR. You can however specify
custom parsing and serializing routines for slots.

After the CLOS class has been annotated using specific slot-options,
objects can be read from an existing XML file (which validates against
the DTD file), or CLOS objects can be serialized to XML. Furthermore,
parent-child relations can be directly created by the parsing code, or
following by the serializing code. XML impex slot options can also be
combined with BKNR indices, which makes browsing XML data very simple.

Exporting and importing XML data now is a breeze!

\section{The BKNR Web Framework}

BKNR also features a web framework, providing an object oriented web
handler architecture. Adding handlers for special kinds of data (for
example RSS feeds for a custom data structure) can easily be achieved
using multiple inheritance. Furthermore, a template handler is
available, making it possible to access LISP functions from an XHTML
file. Sadly, as the web framework has been heavily rewritten in the
last months, we were not able to document it. However, you can have a
look at the example sourcecode for different applications on the bknr
webpage.


%\section{The Templater}


%\section{eboy.com}

